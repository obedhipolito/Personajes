\documentclass{article}
\usepackage[utf8]{inputenc}
\usepackage{graphicx}
\usepackage{ragged2e}

\title{Aplicacion react (Album de personajes)}
\author{Jose Obed Mariano Hipolito }
\date{June 2022}

\begin{document}

\maketitle

\section{Introduction}

\justifying

\textbf{Mi proyecto esta pensado para guardar nuestros personaje favorito que tengamos de un videojuego, pelicula o libro, en el cual se podra a agregar una imagen del personaje y una descipcion que nosotros queramos, por ejemplo un dato importante o interesante del personaje que no queramos olvidar o que nos traiga buenos recuerdos.}

\textbf{La funcion que realiza es similar a un album, con la cualidad de que es en digital y cada usuario puede consultar su lista de personajes agregados para verlos o eliminarlos, con esto el usuario guardaria un dato importante  }

\subsection{Tecnologias usadas}

\justifying{Las tecnologias que se hicieron uso para la realizacion de este proyecto fueron los siguientes:}\\

\justifying{React:Que es la libreria que se ocupo de javascript para poder hacer la aplicacaion web muy similar a lo que puede ser angula pero más facil, que lo podria hacer de un elaboracion más rapida para hacer practicas.}\\

\justifying{Firebase: la base de datos que ayudo a guardar los datos del usuario, el cual se manejo desde la pagina web creada.}\\

\justifying{Autenticacion: hicimos uso de la autentificacion proporcionada por firebase, algo realmente util y que firebase facilita enormemente, la cual ayuda a que cada usuario pueda tener su propio espacio. }\\

\justifying{Storage:hice uso de storage para guardar las imagenes de los personajes que cada usuario agrega, al mismo tiempo este permite visualizarlo a traves de un link }\\

\justifying{Firestore database: esta parte fue de gran importancia ya que aqui recaia el registro de los usuarios en un documento donde se registra todo lo que el usuario realiza.}\\

\subsection{Conclusión}\\

\justifying{Como conclusión puedo decir que realizar este proyecto fue muy interesante y un reto que me ayudo mucho a aprender el uso de las herraminetas antes mencionadas, en el camino me encontre con muchos errores que no conocia y aprendi resolverlos aun que realmente en lo personal me tome mas tiempo de lo que pense. La parte que mas se me dificulto el CRUD, realmente me fue dificil sincronizar bien los datos que el usuario pudiera agregar y se reflejase en firebase, y las reglas que storage tiene para poder almacenar una imagen. Despues de este proyecto me siento mas preparado y entusiasmado de seguir dando froma a las ideas que tenga usando estas tecnologias. Le agaradezco su tiempo por leer este documento}\\









\end{document}
